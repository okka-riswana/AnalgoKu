\section{Merge Sort}

\Problem Buat program Merge-Sort dengan bahasa C++.

\TheSolution Implementasi dapat dilihat di \href{https://github.com/okka-riswana/AnalgoKu/blob/8db63a0d830d90c0f0fba2e2cc53272ffe58b6e7/src/analgoku4/sorting.hpp#L56}{sini}. Berikut adalah sedikit kutipannya,

\begin{code}[cpp]{Merge Sort}
template <typename RandomAccessIterator1,
          typename RandomAccessIterator2,
          typename Compare>
void merge_sort(RandomAccessIterator1 first1,
                RandomAccessIterator1 last1,
                RandomAccessIterator2 first2,
                RandomAccessIterator2 last2,
                Compare comp) {
  const auto distance = std::distance(first1, last1);
  if (distance < 2) {
    return;
  }
  const auto middle1 = std::next(first1, distance >> 1);
  const auto middle2 = std::next(first2, distance >> 1);
  merge_sort(first2, middle2, first1, middle1, comp);
  merge_sort(middle2, last2, middle1, last1, comp);
  Sorting::merge(first2, middle2, middle2, last2, first1, comp);
}
\end{code}

\Problem Kompleksitas waktu algoritma merge sort adalah $O(n \cdot log(n))$. Cari tahu kecepatan komputer anda dalam memproses program. Hitung berapa running time yang dibutuhkan apabila input untuk merge sort-nya adalah 20?

\TheSolution Dari output program , didapat informasi sebagai berikut:
\begin{code}{Output}
Data Size: 20
Result: Sorted
Elapsed: 3367 nanoseconds
\end{code}